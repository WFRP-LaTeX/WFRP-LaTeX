\documentclass[10pt,a4paper]{article}

\usepackage[T1]{fontenc}
\usepackage{libertine}
\usepackage{inconsolata}

\usepackage[top=1.5in,bottom=1in,left=1in,right=1in]{geometry}

\title{WFRP\TeX{} Documentation, Version 1.0.0}
\author{Ian Knight}
\date{}

\begin{document}
\maketitle
\tableofcontents

\section{Introduction}
WFRP\TeX{} is a \LaTeX{} package that comes with two new document classes
designed to make it easier to create WFRP content in \LaTeX{}. This document
will describe how to use the key features of the package to create your
own adventures and source material.


\section{Document classes}
There are two document classes provided by WFRP\TeX{}, for different kinds
of documents. At present, neither class takes any options, so just name the
class like normal. Both classes use exactly the same commands, so just pick
the one that fits your use-case the best.

\paragraph{\texttt{wfrp-long}} The first class is for longer documents that
are divided into chapters, like the Core Rulebook.

\paragraph{\texttt{wfrp-short}} The second class is for shorter documents
that do not have separate chapters, such as shorter adventure supplements.


\section{Commands and Environments}
The main contribution of the package is a number of environments and
commands that simplify the creation of common features of WFRP documents.
These are described in this section.

\subsection{Environments}
Three environments are provided.

\paragraph{\texttt{credits}} The \texttt{credits} environment is used to
create a credits block in your document. If you use one, there should be
only one \texttt{credits} environment in your document, between
\texttt{\textbackslash{}maketitle} and \texttt{\textbackslash{}tableofcontents}.
In \texttt{wfrp-long} this will create a separate page with the credits,
while in \texttt{wfrp-short} the left-hand column on the first page after
the title will contain the credits. This environment will automatically
insert a line crediting WFRP\TeX{}.

\paragraph{\texttt{callout}} Adventures often provide pre-written pieces of
narration to read out to the players. Use the \texttt{callout} environment
to mark these sections. Text inside the environment will be slightly indented
and italicised, but otherwise text will be presented as written, in-line.

\paragraph{\texttt{gmnote}} Often it is useful to provide additional optional
information, either optional rules or notes in an adventure that only apply
to nice situations. The \texttt{gmnote} environment allows the creation of
these rules. The environment takes a single argument which is the note's
title, and then the environment contents are presented in normal text. The
box is shaded grey, and appears in-line.


\end{document}
